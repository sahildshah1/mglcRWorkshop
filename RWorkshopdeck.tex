\documentclass{beamer}
\usetheme{Hannover}
% \usepackage{beamerthemesplit} // Activate for custom appearance
\usepackage{mathtools,parskip,hyperref,courier} 

\title[MGLC \\ R Workshop]{MGLC Transferable Skills Workshop: R}
\author[Sahil Shah]{Sahil Shah \vfill 
	\href{mailto:sahil.shah@u.northwestern.edu}{sahil.shah@u.northwestern.edu}}
\date{May 27, 2015}

\begin{document}

\frame{\titlepage}

\section[Outline]{}
\frame{\tableofcontents}

%%%%%%%%%%%%%%%%%%%%%%%%%%%%%%%%%%%%%%%%%%%%%%%%%
\section{Introduction}
%%%%%%%%%%%%%%%%%%%%%%%%%%%%%%%%%%%%%%%%%%%%%%%%%

% 15 minutes? Introdutions? Ice breakers

\frame{
	\frametitle{Acknowledgments}
	
	\begin{itemize}
		\item McCormick Graduate Leadership Council (MGLC)
		\item Eric Earley, Hayley Belli, Paula Straaton, Bruce Lindvall
	\end{itemize}
	
	Advanced R by Hadley Wickham

}

\frame{
	\frametitle{Schedule}
	
	\begin{itemize}
		\item I'll lecture and you can follow along with my slides
		\item Then we'll break for dinner.
		\item Then you'll reproduce to examples I presented . 
		      You'll pair up ask each and ask me questions 
	\end{itemize}
	
}
	
%%%%%%%%%%%%%%%%%%%%%%%%%%%%%%%%%%%%%%%%%%%%%%%%%
\section{Data Set: Count and Plot, Classification, Regression, Networks}
%%%%%%%%%%%%%%%%%%%%%%%%%%%%%%%%%%%%%%%%%%%%%%%%%

%We'll ask basic questions about this data set to learn the syntax,style etc in R

% Input data. 
% ?, str, etc
% packages...
% console vs. scripts.

%%%%%%%%%%%%%%%%%%%%%%%%%%%%%%%%%%%%%%%%%%%%%%%%%
\section{R Fundamentals}
%%%%%%%%%%%%%%%%%%%%%%%%%%%%%%%%%%%%%%%%%%%%%%%%%

\frame{
	\frametitle{Vocab}
	
	\begin{itemize}
	
		\item \texttt{?}
		\item \texttt{str()} gives short description of any R data structure
		\item \texttt{ls}
	
	\end{itemize}

}

\frame{
	\frametitle{Importing Data}
	
	\texttt{read.table}
	
	header	
	a logical value indicating whether the file contains the names of the variables as its first line.




}




\frame{
	\frametitle{Data types}
	
	\begin{itemize}
		\item R has no 0-dimensional, or scalar types. 
		\item Individual numbers or strings are actually vectors of length one
	\end{itemize}
	
	\begin{itemize}
		\item Logical
		\item Integer
		\item Double (aka Numeric)
		\item Character
	\end{itemize}
	
}

\frame{
	\frametitle{Data structures}
	
	Homogenous data structures
	\begin{itemize}
		\item[1d] Atomic vector
		\item[2d] Matrix
		\item[nd] Array
	\end{itemize}
	
	Heterogeneous data structures
	\begin{itemize}
		\item[1d] List
		\item[2d] Data frame
	\end{itemize}
	
}

\frame{
	\frametitle{Subsetting}
	
	\texttt{$}



}


\frame{
	\frametitle{Functions}


}


\frame{
	\frametitle{Functionals}
	
	`Split-Combine-Apply'
	
	lapply

}

\frame{

	\frametitle{Loops}
	
	\begin{itemize}
		\item
		\item \texttt{while}
		
	\end{itemize}
	
}

\frame{
	\frametitle{Plotting}


}

\frame{
	\frametitle{Style}
	
	Use a consistent style
	
	\begin{itemize}
		\item Strive for names that are concise and meaningful 
		\item Use \texttt{<-} not \texttt{=} for assignment
		\item Comment with \texttt{\#}
		
	\end{itemize}

}

\frame{
	\frametitle{Sweave}




}

%%%%%%%%%%%%%%%%%%%%%%%%%%%%%%%%%%%%%%%%%%%%%%%%%
\section{Hands-On Practice}
%%%%%%%%%%%%%%%%%%%%%%%%%%%%%%%%%%%%%%%%%%%%%%%%%

% Put students into group like in Sarah Iams recitations 

\frame{
	\frametitle{References}
	
	\begin{itemize}
		\item \url{http://adv-r.had.co.nz/}
		\item \url{http://google-styleguide.googlecode.com/svn/trunk/Rguide.xml}
		\item \url{http://www.statmethods.net/}
		\item \url{http://nicercode.github.io/}
		\item \url{http://stackoverflow.com/}
		\item \url{http://www.statistik.tuwien.ac.at/public/filz/students/SweaveExa.pdf}
		\item \url{http://www.stat.columbia.edu/~tzheng/files/Rcolor.pdf}
		
	\end{itemize}
	
}

\end{document}
